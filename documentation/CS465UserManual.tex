%%%%%%%%%%%%%%%%%%%%%%%%%%%%%%%%%%%%%%%%%
% Short Sectioned Assignment
%
% Template's Original author:
% Frits Wenneker (http://www.howtotex.com)
%
% License:
% CC BY-NC-SA 3.0 (http://creativecommons.org/licenses/by-nc-sa/3.0/)
%
%%%%%%%%%%%%%%%%%%%%%%%%%%%%%%%%%%%%%%%%%

%----------------------------------------------------------------------------------------
%	PACKAGES AND OTHER DOCUMENT CONFIGURATIONS
%----------------------------------------------------------------------------------------

\documentclass[paper=a4, fontsize=11pt]{scrartcl} % A4 paper and 11pt font size

\usepackage[T1]{fontenc} % Use 8-bit encoding that has 256 glyphs
%\usepackage{fourier} % Use the Adobe Utopia font for the document - comment this line to return to the LaTeX default
\usepackage[english]{babel} % English language/hyphenation
\usepackage{amsmath,amsfonts,amsthm} % Math packages

\usepackage{lipsum} % Used for inserting dummy 'Lorem ipsum' text into the template
\usepackage{hyperref}

\usepackage{sectsty} % Allows customizing section commands
\allsectionsfont{\centering \normalfont\scshape} % Make all sections centered, the default font and small caps

\usepackage{fancyhdr} % Custom headers and footers
\pagestyle{fancyplain} % Makes all pages in the document conform to the custom headers and footers
\fancyhead{} % No page header - if you want one, create it in the same way as the footers below
\fancyfoot[L]{} % Empty left footer
\fancyfoot[C]{} % Empty center footer
\fancyfoot[R]{\thepage} % Page numbering for right footer
\renewcommand{\headrulewidth}{0pt} % Remove header underlines
\renewcommand{\footrulewidth}{0pt} % Remove footer underlines
\setlength{\headheight}{13.6pt} % Customize the height of the header

\numberwithin{equation}{section} % Number equations within sections (i.e. 1.1, 1.2, 2.1, 2.2 instead of 1, 2, 3, 4)
\numberwithin{figure}{section} % Number figures within sections (i.e. 1.1, 1.2, 2.1, 2.2 instead of 1, 2, 3, 4)
\numberwithin{table}{section} % Number tables within sections (i.e. 1.1, 1.2, 2.1, 2.2 instead of 1, 2, 3, 4)

\setlength\parindent{0pt} % Removes all indentation from paragraphs - comment this line for an assignment with lots of text

%----------------------------------------------------------------------------------------
%	TITLE SECTION
%----------------------------------------------------------------------------------------

\newcommand{\horrule}[1]{\rule{\linewidth}{#1}} % Create horizontal rule command with 1 argument of height

\title{	
\normalfont \normalsize 
\textsc{CS 465} \\ [25pt] % Your university, school and/or department name(s)
\horrule{0.5pt} \\[0.4cm] % Thin top horizontal rule
\huge Virus Project User Manual \\ % The assignment title
\horrule{2pt} \\[0.5cm] % Thick bottom horizontal rule
}

\author{Caleb Dingus, Emmanuel Onwuka, Mardigon Toler} % Your name

\date{May 2, 2017} % Today's date or a custom date

\begin{document}

\maketitle % Print the title

\section{Installing the Software}

To download the Windows binaries for the fake login client and the malicious modification of jEdit, visit:
\url{https://github.com/mardiqwop/cs465project/releases/tag/0.2}

\vspace{5mm}

The archived files "jedit.zip" and "SteamPhish.zip" contain everything needed to run the malicious jEdit modification and the login client, respectively. 
This is everything you need to run the viruses themselves. Instructions for running your own instance of this server are described below. 
\vspace{5mm}

The server, whose source code can be found on that GitHub project, is a script written in Python 3. To run your own server for this project,
you must modify the source code file DataHandler.java for both the login client and jEdit such that the socket connection will bind to the IP address of your own server. To build jEdit with these modifications, you must obtain a copy of
its source code from its website and replace some of its source files with the versions included in our GitHub repository.
When running your server with Python 3, the data sent to the server is, by default, stored in a log file called captured.txt.

\section{Removing the Software}
The software can be removed from your machine simply by deleting the directory that contains it.
After it is uninstalled, some desktop shortcuts may be incorrect. Check their targets to make sure they are shortcuts for trusted software.


\end{document}